\documentclass{article}
\usepackage{graphicx}
\usepackage{float}
\usepackage[dvipsnames]{xcolor}
\usepackage{listings}
\usepackage{indentfirst}
\usepackage{paralist}

\begin{document}
%\lstset{numbers=left,numberstyle=\tiny,keywordstyle=\color{blue!70},commentstyle=\color{red!50!green!50!blue!50},frame=shadowbox, rulesepcolor=\color{red!20!green!20!blue!20},escapeinside=``,xleftmargin=2em,xrightmargin=2em, aboveskip=1em}
\lstset{numbers=left,framexleftmargin=10mm,frame=none,backgroundcolor=\color[RGB]{245,245,244},keywordstyle=\bf\color{blue},identifierstyle=\bf,numberstyle=\color[RGB]{0,192,192},commentstyle=\it\color[RGB]{0,96,96},stringstyle=\rmfamily\slshape\color[RGB]{128,0,0},showstringspaces=false}
\title{\textbf{Blivet analysis}}
\author{Tao Wu}
\date{\today}
\maketitle


\section{Introduction}
Blivet provides an easy access to disk management, including partitioning on
disks, making format such as ext4,ext3 or xfs on partitions, creating pv/vg/lv
and so on.  This article focus on the internal implement of blivet, which will
help users to locate source of bugs more quickly.
  
  A normal work flow of blivet is as follows:
\begin{itemize}
\item Collect device information from udev and build a device tree
\item Get info from device tree or change device tree and create actions
\item Use partition algorithm to do partition inside device tree
\item Process actions in device tree so that changes can be written into disks
\end{itemize}

\section{Build device tree}

The device tree is created through populate() method:
1. set maximal waiting time to 300 secends for udev to be done.
2. addUdevDevice
     add udevXXXdevice
3. handleUdevDeviceFormat
     handleUdevXXXFormat
\subsection{libudev intro}
\subsection{devices intro}
\subsection{formats intro}
\subsection{devicetree structure}
get devices:\\
    def getDeviceBySysfsPath(self, path, incomplete=False, hidden=False):\\
    def getDeviceByUuid(self, uuid, incomplete=False, hidden=False):\\
    def getDeviceByLabel(self, label, incomplete=False, hidden=False):\\
    def getDeviceByName(self, name, incomplete=False, hidden=False):\\
    def getDeviceByPath(self, path, preferLeaves=True, incomplete=False, hidden=False):\\
    def getDeviceByID(self, id\_num, hidden=False):\\
    def getDevicesByType(self, device\_type):\\
    def getDevicesByInstance(self, device\_class):\\
    def getDevicesBySerial(self, serial, incomplete=False, hidden=False):\\
\\
add device:\\
    def addUdevLVDevice(self, info):\\
    def addUdevDMDevice(self, info):\\
    def addUdevMultiPathDevice(self, info):\\
    def addUdevMDDevice(self, info):\\
    def addUdevPartitionDevice(self, info, disk=None):\\
    def addUdevDiskDevice(self, info):\\
    def addUdevOpticalDevice(self, info):\\
    def addUdevLoopDevice(self, info):\\
    def addUdevDevice(self, info):\\
\\
handle format:\\
    def handleUdevDeviceFormat(self, info, device):\\
    def handleUdevDiskLabelFormat(self, info, device):\\
    def handleUdevLVMPVFormat(self, info, device):\\
    def handleVgLvs(self, vg\_device):\\
    def handleUdevLUKSFormat(self, info, device):\\
    def handleUdevMDMemberFormat(self, info, device):\\
    def handleUdevDMRaidMemberFormat(self, info, device):\\
    def handleBTRFSFormat(self, info, device):\\
\\
actions:\\
    def pruneActions(self):\\
    def sortActions(self):\\
    def processActions(self, dryRun=None):\\
    def registerAction(self, action):\\
    def cancelAction(self, action):\\
    def findActions(self, device=None, type=None, object=None, path=None,\\
\\
other:\\
    def devices(self):\\
    def filesystems(self):\\
    def uuids(self):\\
    def labels(self):\\
    def leaves(self):\\



\section{Do partition}

Blivet use blivet.partitioning.doPartitioning() method to (re)configure all the
partition requests so that each of them can get a proper location on the disks.
\subsection{sort requests}
\subsection{remove new partitions}
\subsection{allocate partitions}
\subsection{grow partitions}
\subsection{manageSizeSet}


\section{Write changes to disks}
Blivet.doIt()
devicetree.processActions()<++>
pruneActions()<++>
sortActions()

sorting actions using tsort

\section{Device factory}

in _init:
facortyDevice(devicetype,size,kwargs)
  get device factory
  factory.configure
return factory.device

factory(blivet,size,disks,kwargs)

configure():
save device tree
try: _configure
if fail: revert devicetree

_configure:
set container
  get container(device,container_name)
     if device has a vg attribute:
          container = this vg
     elif has name
        container_name is a vg in blivet.vgs
          container = this vg(container_name)
     else no device no name
        if can get a list of all vgs in blivet.vg and it is not exist
        get the biggest/smallest vg
        container = this vg
        else
          container = None
setup child factory:
  get args
    get total space
    get disks
  get kwargs
	get child factory fstype -defined in the class
child factory.config() -reconfigure/create several pvs
  #this might change container's disks(pvs) only if we adjust a device
reconfigure/create container  -VG
    adjust container to make it meet the child factory
reconfigure/create device  -LV

#pv just create once, and then can be resized by the totalsizeset when
doPartitioning.


\section{Other issue}
1. When using recursiveRemove() on one disk, it will not remove the request on
that disk, which may cause a bit choas. For example, you create two partitions
on disk sdc without doIt, and then you decide to remove sdc, but you will find
you have the two partitions on sdc created after you excute doIt, which is not
what you really want.
2. Partition algorithm.
3. ManageSizeSet-float.
3. Multipath discovery.
4. self._reconfigure_container in devicefactory.py seems do nothing.
5. handle_no_size in device factory seems useless.

\end{document}
