\documentclass{article}
\usepackage{graphicx}
\usepackage{float}
\usepackage[dvipsnames]{xcolor}
\usepackage{listings}
\usepackage{indentfirst}
\usepackage{paralist}

\begin{document}
%\lstset{numbers=left,numberstyle=\tiny,keywordstyle=\color{blue!70},commentstyle=\color{red!50!green!50!blue!50},frame=shadowbox, rulesepcolor=\color{red!20!green!20!blue!20},escapeinside=``,xleftmargin=2em,xrightmargin=2em, aboveskip=1em}
\lstset{numbers=left,framexleftmargin=10mm,frame=none,backgroundcolor=\color[RGB]{245,245,244},keywordstyle=\bf\color{blue},identifierstyle=\bf,numberstyle=\color[RGB]{0,192,192},commentstyle=\it\color[RGB]{0,96,96},stringstyle=\rmfamily\slshape\color[RGB]{128,0,0},showstringspaces=false}
\title{\textbf{Blivet investigation report}}
%\title{\textbf{How to use blivet on LPAR}}
\author{Tao Wu}
\date{\today}
\maketitle
\section{Introduction}

Blivet is a python module for system storage configuration.\\

A good feature of blivet is that we can use it to model a series of changes
without necessarily committing any of the changes to disk. We can schedule an
arbitrarily large series of changes, seeing the effects of each as it is
scheduled. Nothing is written to disk, however, until you execute the actions.\\

Version: 0.18.34 \\

Installation:  yum install python-blivet libselinux-python\\
\section{What does blivet provide}
\begin{tabular}{|l|c|r|}
\hline
Function & Support & Remark\\
\hline
List devices & Y & \\
Delete devices & Y & \\
Partition-manually & Y & \\
Partition-automation & Y & Not verified\\
Format & Y & \\
Mount & Y & \\
Unmount & Y & \\
Discover ZFCP/SCSI disks & Y & \\
Create PV & Y & \\
Create VG & Y & \\
Create LV & Y & \\
Resize & Y & Not verified\\
Encrypt & Y & Not verified\\
\hline
\end{tabular}\\
\clearpage

Arch supported:\\
\newline
\begin{tabular}{|l|c|}
\hline
    Cell & Y\\
    Mactel & Y\\
    Efi & Y\\
    \textcolor{red}{X86} & Y\\
    PPC & Y\\
    \textcolor{red}{S390} & Y\\
    IA64 & Y\\
    Alpha & Y\\
    Sparc & Y\\
    AARCH64 & Y\\
    ARM & Y\\
\hline
\end{tabular}\\
\\
\\


Filesystem supported:\\
\newline
\begin{tabular}{|l|c|}
\hline
Filesystem & Support \\
\hline
    \textcolor{red}{Ext2FS} & Y\\
    \textcolor{red}{Ext3FS} & Y\\
    \textcolor{red}{Ext4FS}& Y\\
    FATFS & Y\\
    EFIFS & Y\\
    \textcolor{red}{BTRFS} & Y\\
    GFS2 & Y\\
    JFS & Y\\
    ReiserFS & Y\\
    \textcolor{red}{XFS} & Y\\
    HFS & Y\\
    AppleBootstrapFS & Y\\
    HFSPlus & Y\\
    NTFS & Y\\
    NFS & Y\\
    NFSv4 & Y\\
    Swap & Y\\
    Iso9660FS & Y\\
    NoDevFS & Y\\
    DevPtsFS & Y\\
    ProcFS & Y\\
    SysFS & Y\\
    TmpFS & Y\\
    BindFS & Y\\
    SELinuxFS & Y\\
    USBFS & Y\\
\hline
\end{tabular}

\section{Basic methods we may need}
Blivet supplies a great many methods to manipulate storage devices recognized
by kernel. In the first instance, I think we should focus on the following ones
which are sufficient enough to implement the functions refereed in section two.
\begin{itemize}
\item \textbf{blivet.Blivet.reset()} \\-Reset storage configuration to reflect
actual system state.\\
\item \textbf{blivet.devicetree.DeviceTree.getDeviceByName(name, incomplete=False, hidden=False)}\\-Return a device
with a matching name.\\
    \textbf{\textbf{Parameters:}}	\\
        name (str) - the name to look for\\
        incomplete (bool) - include incomplete devices in search\\
        hidden (bool) - include hidden devices in search\\
    \textbf{Returns:}	the first matching device found\\
    \textbf{Return type:}	device\\
\item \textbf{blivet.Blivet.recursiveRemove(device)}\\-Remove a device after removing its
dependent devices.\\
We use this method to destroy devices which are not leaves.\\
\item \textbf{blivet.Blivet.initializeDisk(disk)}\\-(Re)initialize a disk by creating a disklabel on it.\\
\textbf{Parameters:}	\\disk (StorageDevice) - the disk to initialize\\
\textbf{Returns:} None\\
\textbf{Raises:}	ValueError\\
\item \textbf{blivet.Blivet.newPartition(*args, **kwargs)}\\-Return a new (unallocated) PartitionDevice
instance.\\
       \textbf{Parameters:}	\\
       fmt\_type (str) - format type\\
       fmt\_args (dict) - arguments for format constructor\\
       mountpoint (str) - mountpoint for format (filesystem)\\
\item \textbf{blivet.Blivet.createDevice(device)}\\ -Schedule creation of a device.\\
    \textbf{Parameters:}\\	device (StorageDevice) - the device to schedule creation of\\
    \textbf{Return type:}	None\\
\item \textbf{blivet.formats.getFormat(fmt\_type, *args, **kwargs)}\\- Return an
instance of the appropriate DeviceFormat class.\\
    \textbf{Parameters:}	\\fmt\_type (str) - The name of the formatting type\\
    \textbf{Returns:}	the format instance\\
    \textbf{Return type:}	DeviceFormat\\
    \textbf{Raises:}	ValueError\\
\item \textbf{blivet.partitioning.doPartitioning(storage)}\\-Allocate and grow
partitions.\\
    \textbf{Parameters:}	\\storage (Blivet) - Blivet instance\\
    \textbf{Raises:}	PartitioningError\\
    \textbf{Returns:}	None\\
\item \textbf{blivet.Blivet.formatDevice(device, fmt)}\\-Schedule formatting of a device.\\
    \textbf{Parameters:}	\\
        device (StorageDevice) - the device to create the formatting on\\
        fmt - the format to create on the device\\
    \textbf{Return type:}	None\\
    A format destroy action will be scheduled first, so it is not necessary to
create and schedule an ActionDestroyFormat prior to calling this method.\\
\item \textbf{blivet.Blivet.newVG(*args, **kwargs)}\\- Return a new LVMVolumeGroupDevice
instance.\\
	\textbf{Parameters:}	\\
    name (str) - the device name (generally a device node’s basename)\\
    exists (bool) - does this device exist\\
    parents (list of StorageDevice) – a list of parent devices\\
    sysfsPath (str) - sysfs device path\\
    peSize (Size) - physical extent size\\
    \textbf{Returns:}	the new volume group device\\
    \textbf{Return type:}	LVMVolumeGroupDevice\\
\item \textbf{blivet.Blivet.newLV(*args, **kwargs)}\\- Return a new LVMLogicalVolumeDevice
instance.\\
    \textbf{Parameters:}	\\
        fmt\_type (str) - format type\\
        fmt\_args (dict) - arguments for format constructor\\
        mountpoint (str) - mountpoint for format (filesystem)\\
        thin\_pool (bool) - whether to create a thin pool\\
        thin\_volume (bool) - whether to create a thin volume\\
    \textbf{Returns:}	the new device\\
    \textbf{Return type:}	LVMLogicalVolumeDevice\\
\item \textbf{blivet.Blivet.doIt()}\\-Commit queued changes to disk.\\-Caution: Data on
devices may be destroyed after this method succeeds.\\
\item \textbf{blivet.zfcp.ZFCP.addFCP(devnum, wwpn, fcplun)}\\-Manully online ZFCP/SCSI disks.\\
    \textbf{Parameters:}	\\
        devnum (str) - scsi disk number\\
        wwpn (str) - world wide port name\\
        fcplun (str) - logical unit number\\
    \textbf{Return type:}	None\\
\item \textbf{blivet.devicetree.populate(cleanupOnly=False)}\\- Locate all storage devices.\\
Loop and scan for devices to build the device tree.\\

\end{itemize}

\section{A simple example on LPAR11}

\begin{lstlisting}[language=python]
import blivet

b=blivet.Blivet()
b.reset()

#initialize disk
dasdb=b.devicetree.getDeviceByName("dasdb")
b.recursiveRemove(dasdb)
b.initializeDisk(dasdb)

#create new PV
part=b.newPartition(size=8000,parents=[dasdb])
b.createDevice(part)
fmt=blivet.formats.getFormat("lvmpv",lable="pv1")
b.formatDevice(part,fmt)
blivet.partitioning.doPartitioning(b)
b.doIt()

# create new VG
a=b.pvs
request=b.newVG(parents=a)
b.createDevice(request)
b.doIt()

#create new LV
a=b.vgs
request=b.newLV(size=2500,name="lv1",parents=a)
b.createDevice(request)
fmt=blivet.formats.getFormat("ext4",lable="test1",
    mountpoint="/home/test1")
b.formatDevice(request,fmt)
blivet.partitioning.doPartitioning(b)
b.doIt()

fmt.mount()

#discover scsi devices
a=b.zfcp
a.addFCP("0.0.5080","0x500507630300c52a",
		"0x4013401300000000")
b.devicetree.populate()
\end{lstlisting}
\section{Discovery of ZFCP disks}
To manually online ZFCP disks, first of all we invoke the addFCP() methods refered
in Section three. We can also put these arguments in a configure file as anaconda
does. Then we use populate() to add the ZFCP disk to devicetree so that blivet
can see it.\\
\newline
In order to get the arguments needed, we can use some tools such as: \\
udevadm info -a -p /sys/class/scsi\_generic/sg0
\section{What to do in the next stage}
\begin{enumerate}
\item Do as much tests as more to make sure blivet work smoothly on all kinds of
disk status (encrypted?  mounted? have some children devices? have some special
data on it being used?  etc.)
\item Continue to explore blivet codes and dig out some useful methods.
\end{enumerate}
\end{document}
